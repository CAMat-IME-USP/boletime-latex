\section*{Novo projeto acadêmico para o próximo ciclo}

O projeto acadêmico é um documento estratégico, norteando as ações do instituto e departamentos dentro de um ciclo. Como dispõe de uma análise conjuntural da unidade e estipula metas a serem alcançadas através de atitudes pontuadas como necessárias, o projeto acadêmico permite, ao final do ciclo, a realização de um balanço da atuação institucional e docente naquele período. Ou seja, este documento não é meramente técnico, mas político.

A USP está no momento de reformulação do projeto acadêmico das unidades - e o IME USP não é isolado disso. Na semana, o representante discente na Comissão de Graduação recebeu o rascunho inicial - o qual iremos disponibilizar ao lado - das metas sugeridas para o próximo ciclo de 5 anos do instituto, tendo o prazo de 9 de junho (segunda-feira) para apresentar propostas de alteração ou supressão. Posteriormente, o projeto consolidado na CG passará para a Congregação, onde novamente teremos possibilidade de intervenção.

Fazemos, então, um chamado à comunidade IMEana para realizarmos um balanço das metas postas no ciclo anterior e o quanto, na visão dos estudantes, o instituto avançou ou não. E mais, colocarmos as nossas ponderações, propostas e críticas às metas rascunhadas para o próximo ciclo. 

\quadradao{\LARGE}{1.2}{0}{\columnwidth}{black}{0.15cm}{
REUNIÃO ABERTA
\\~\\
Revisão do Projeto Acadêmico
\\~\\
06 de junho (quinta-feira) \\ às 17h | Saguão do Bloco B
\\~\\
}

% para ir para a outra coluna
\vfill\null\columnbreak

\tituloCentralizado{Formulário para Colaboração}
\begin{figure}[H]
    \centering
    \includegraphics[width=0.5\linewidth]{textos//img/qrcode_colaboracao.png}
\end{figure}

\vspace{1cm}

\tituloCentralizado{Rascunho das metas - CG}
\begin{figure}[H]
    \centering
    \includegraphics[width=0.5\linewidth]{textos//img/qrcode_rascunho_metas.png}
\end{figure}

\vspace{1cm}

\tituloCentralizado{Projeto Acadêmico vigente}
\begin{figure}[H]
    \centering
    \includegraphics[width=0.5\linewidth]{textos//img/qrcode_projeto_academico_vigente.png}
\end{figure}