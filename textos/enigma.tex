\section*{Enigma do BoletIME - Resolva e ganhe 1 Trento}

A primeira e a décima pessoa a apresentarem a solução correta do enigma abaixo para o CAMat (seja pelo envio no e-mail ou pessoalmente para algum membro da gestão) ganhará um Trento da lojinha do CAMat. Boa sorte!

Um professor do IME decidiu, após 4 anos lecionando, contabilizar quantas avaliações corrigira. O professor deu aula em todos os períodos para apenas uma turma, que sempre era composta de 40 alunos. Para avaliar seus estudantes, ele aplicou 2 provas durante o semestre, não ofereceu substitutivas e tem um sistema de atividades para que a pessoa escolha, dentre as quatro listas oferecidas, duas para serem feitas e corrigidas. Ele constatou que, em cada uma das suas avaliações (provas e atividades), exatamente 9 alunos estavam ausentes ou simplesmente não a entregaram. Curiosamente, o professor também sabe que apenas 9 alunos já fizeram a prova de recuperação.

O professor decidiu, então, conferir no sistema quantas avaliações haviam sido feitas, mas, infelizmente, o sistema apresentou falhas no armazenamento. Sabe-se que o sistema registra a quantidade de notas numa base 10, mas a falha acabou transformando os algarismos em letras de modo que letras iguais representam números iguais. Ademais, a falha também acabou multiplicando a quantidade de avaliações pelo número de estudantes que não compareceu às provas acrescido do dia do mês em que a consulta foi feita.

Determine uma combinação de letras que o sistema pode ter apresentado quando consultado após as falhas.