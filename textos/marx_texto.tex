\section*{Efeitos da redução da jornada de trabalho sobre os salários}
\autoria{Karl Marx}

[...]

 Todos vocês conhecem a Lei das Dez Horas ou, mais precisamente, das Dez Horas e Meia, promulgada em 1848 [na Inglaterra]. Foi uma das maiores modificações econômicas que já presenciamos. Representou um aumento súbito e obrigatório de salários não em umas poucas indústrias locais, mas nos ramos industriais mais eminentes, aqueles por meio dos quais a Inglaterra domina os mercados do mundo. Foi uma alta de salários em circunstâncias singularmente desfavoráveis. O dr. Ure, o prof. Senior e todos os demais porta-vozes oficiais da burguesia no campo da economia demonstraram (e devo dizer, com razões muito mais sólidas do que as do nosso amigo Weston), que aquilo seria o “dobre de finados” [o ato de soar os sinos pela morte de alguém] da indústria inglesa. Demonstraram que não se tratava de um simples aumento de salário, mas de um aumento de salários provocado pela redução da quantidade de trabalho empregado, e nela fundamentado. Afirmaram que a décima segunda hora que se queria arrancar dos capitalistas era justamente aquela na qual eles obtinham o seu lucro. Ameaçaram com o decréscimo da acumulação, a alta dos preços, a perda dos mercados, a redução da produção, a consequente reação sobre os salários e, enfim, a ruína. Sustentavam que a lei de Maximilian Robespierre sobre os limites máximos [dos preços de mercadorias e salários] era uma ninharia comparada com esta outra; e, até certo ponto, tinham razão. Mas qual foi, na realidade, o resultado?

Os salários em dinheiro dos operários fabris aumentaram, apesar de haver reduzido a jornada de trabalho; cresceu consideravelmente o número de operários em atividade nas fábricas; baixaram constantemente os preços dos seus produtos; desenvolveram-se às mil maravilhas as forças produtivas do seu trabalho e se expandiram progressivamente, em proporções nunca vistas, os mercados para os seus artigos. Em Manchester, na assembleia da Sociedade Pelo Progresso da Ciência, em 1860, eu próprio ouvi o sr. Newman confessar que ele, o dr. Ure, Senior e todos os demais representantes oficiais da ciência econômica se haviam equivocado, ao passo que o instinto do povo não falhara. Cito neste passo o sr. W. Newman e não o prof. Francis Newman, porque ele ocupa na ciência econômica um lugar proeminente, como colaborador e editor da “História dos Preços”, da autoria do sr. Thomas Tooke, essa obra magnífica, que remonta a história dos preços desde 1793 a 1856. Se estive correta a ideia fixa de nosso amigo Weston sobre o volume fixo dos salários de um volume de produção fixo, de um grau fixo de produtividade do trabalho, de uma vontade fixa o constante dos capitalistas e tudo o mais que há de fixo e imutável em Weston, então o prof. Senior teria acertado em seus sombrios presságios, e Robert Owen teria se equivocado, ele que, já em 1816, pedia uma limitação geral da jornada de trabalho como primeiro passo preparatório para a emancipação da classe operária, implantando-a efetivamente, por conta e risco próprios, na sua fábrica têxtil de New Nanark, contra o preconceito generalizados.

[...]