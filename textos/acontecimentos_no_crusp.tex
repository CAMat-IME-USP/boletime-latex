\section*{Acontecimentos no CRUSP}
{\color{vermelho_camat} \textbf{Nota conjunta sobre a situação do CRUSP}}
\autoria{14 de agosto de 2024 \\Centro Acadêmico da Matemática, Estatística e\\Computação\\Centro Acadêmico da Física\\Centro Acadêmico Favo 22}


Pela manhã desta quarta-feira (14/08), houveram tensionamentos nos blocos F e G do CRUSP, após a Pró-Reitoria de Inclusão e Pertencimento (PRIP) dar início ao processo de instalação de grades, cujo amplo anúncio havia sido realizado por volta das 7h de hoje (14/08). Vale ressaltar que nesta manhã também estava ocorrendo o ato pelo Dia Nacional dos Estudantes, na Avenida Paulista, não sendo coincidência o início desse processo logo no dia e hora em que parcela dos estudantes mobilizados estariam afastados da universidade.

Dois membros do CAMat estiveram no local, e relatam que a maior parte dos tensionamentos ocorreram no bloco G, onde as instalações estavam mais avançadas, sendo palco para embate entre moradores favoráveis às grades (ligados a gestão anterior da AMORCRUSP) e moradores contrários às grades, ainda que favoráveis a aprimoração de medidas de segurança no CRUSP. As instalações, em ambos os blocos, foram suspensas por hoje após negociação da AMORCRUSP com uma representante da PRIP. Ainda, a associação convoca os moradores para uma assembleia extraordinária, às 17h, para debater os próximos passos frente a essa ofensiva, e endossamos o chamado a todos estudantes do IF, IME e CM e moradores do CRUSP!

Em nota, PRIP publicou um vídeo pelo Instagram colocando a instalação das grades como uma demanda histórica dos moradores, alegando AMORCRUSP (Associação de Moradores do CRUSP) de espalhar desinformação quanto à legitimidade e aprovação dos moradores da medida, uma vez que a instalação foi debatida “com presença da representação estudantil, reunindo quase uma centena de pessoas”. Importante notar a semelhança desses espaços com as audiências sobre a reformulação do PAPFE (vide BoletIME \#3), em que aos estudantes foi dado o poder da voz, mas não o poder de votar e definir concretamente os rumos da política de permanência.

Os CAs signatários entendem que se trata de um assunto delicado e polêmico, dividindo opiniões entre os próprios moradores, sendo um assunto com particularidades e nuances. O CRUSP não é isento de problemas de segurança. De fato, são muitos, e nisso, é necessário compreender que ser contra a instalação de portões e grades não é sinônimo de ser contra um espaço seguro de moradia, como a PRIP insinua dizendo: “A quem interessa ir contra a segurança no CRUSP?”. Ainda, é importante colocar que portões não resolvem os problemas de segurança pela causa, que é um debate necessário e importante, e igualar essas medidas de controle de acesso com segurança da moradia estudantil é contornar esse debate, retirando o cerne político e rebaixando a um debate técnico.

Por isso, reforçamos novamente o chamado aos moradores para comparecerem na assembleia, às 17h30min, e estendemos o chamado para todos estudantes se apropriarem do que vem acontecendo, formulando em conjunto e atuando ativamente, em conjunto dos moradores, nessa luta por moradia digna e segura!

\quadradao{\normalsize}{1.2}{0}{\columnwidth}{black}{0.15cm}{
{\Large Participe ativamente da construção de um CRUSP popular!}
\\~\\
\color{black} \justifying \normalfont % para ignorar o bold
A fim de organizar aqueles interessados em compor a luta por um CRUSP melhor e popular, foi feito o formulário abaixo com objetivo de reunir moradores e simpatizantes que desejam participar ativamente dessa luta.
\\~\\
Queremos construir um espaço de moradia que seja seguro e inclusivo, onde as políticas de segurança sejam formuladas em conjunto com os estudantes que aqui residem.
\\
\begin{figure}[H]
    \centering
    \includegraphics[width=3cm]{textos//img/crusp_qr_code.png}
    \href{https://forms.gle/8vkN32mbbG5rP9q4A}{https://forms.gle/8vkN32mbbG5rP9q4A}
\end{figure}
}
